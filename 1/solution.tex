\documentclass{exam}
\usepackage{wrapfig}
\usepackage{tikz}
\usepackage[a4paper,bindingoffset=0 cm, left=1.8cm, right=1.8cm, top=1.8cm, bottom=1.8cm, footskip=0.5cm]{geometry}
\usepackage{amssymb,amsmath}
\usepackage[utf8]{inputenc}
\usepackage[russian]{babel}
\setlength{\parindent}{0em}
\setlength{\parskip}{1em}
\usepackage[fontsize=12pt]{scrextend}
\usepackage{graphicx}
\usepackage{float}

\begin{document}
    
\textbf{Прогрессивная последовательность, прогрессивная ли?}

\textbf{Условие:} Из множества $\{1,2,\dots,97\}$ выбирают три числа. Какова вероятность, что из них можно составить арифметическую прогрессию?

\textbf{Ответ:} $\frac{144}{9215} \approx 0.015627$

\textbf{Решение:} Для того, чтобы выбрать три числа составляющую арифметическую прогрессию, достаточно выбрать два числа одинаковой четности, которые будут наибольшим и наименьшим членами этой прогресcии.

Следовательно, искомая вероятность равна сумме количества способов выбрать два четных числа и количеству способов выбрать два нечетных числа из множества $\{1,2,\dots,97\}$ поделенную на общее количество способов выбрать три числа из этого же множества, т. е.

$$\frac{C_{48}^{2} + C_{49}^{2}}{C_{97}^{3}} = \frac{144}{9215}$$

\end{document}