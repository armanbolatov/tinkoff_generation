\documentclass{exam}
\usepackage{wrapfig}
\usepackage{tikz}
\usepackage[a4paper,bindingoffset=0 cm, left=1.8cm, right=1.8cm, top=1.8cm, bottom=1.8cm, footskip=0.5cm]{geometry}
\usepackage{amssymb,amsmath}
\usepackage[utf8]{inputenc}
\usepackage[russian]{babel}
\setlength{\parindent}{0em}
\setlength{\parskip}{1em}
\usepackage[fontsize=12pt]{scrextend}
\usepackage{graphicx}
\usepackage{float}

\begin{document}
    
\textbf{ВКвадратное уравнение}

\textbf{Условие:} Вася пишет функцию $f(x) = x^2 + bx + c$, причем коэффициенты $b, c$ он выбирает наугад из квадрата с вершинами, лежащими в точках $(2; 2), (-2; 2), (2; -2), (2, 2)$. Найдите вероятность того, что корни окажутся мнимыми.

\textbf{Ответ:} $\frac{5}{12}$.

\textbf{Решение:} Чтобы корни Васиного квадратного трехчлена оказались мнимыми, необходимо и достаточно, чтобы его дискриминант оказался отрицательным, т. е.
\begin{equation}\label{star}
    D(f) = b^2 - 4c < 0 \iff c > \frac{b^2}{4}. \tag{$*$}
\end{equation}

\begin{wrapfigure}{r}{0pt}
    \hspace{10pt}
    \begin{tikzpicture}
        \draw[fill=blue!20!white, draw=blue, thick] plot[smooth,samples=100,domain=-2:2, blue] ({\x}, {0.25*\x*\x}) -- 
        plot[smooth,samples=100,domain=2:-2] (\x,{2});
        \draw[color=gray,step=1cm,dashed] (-3,-3) grid (3,3);
        \draw[thick, ->] (-3, 0) -- (3, 0) node[right] {$x$};
        \draw[thick, ->] (0, -3) -- (0, 3) node[above] {$y$};
        \draw[thick, red] (-2, -2) rectangle (2, 2);
    \end{tikzpicture}
\end{wrapfigure}

Начертим график $g(x) = \frac{x^2}{4}$ синим цветом на рисунке справа. Понятно, что все точки $(b, c)$ удовлетворяющие (\ref{star}) должны лежать выше данного графика. А так как Вася выбирает внутри квадрата с вершинами, лежащими в точках $(2; 2), (-2; 2), (2; -2), (2, 2)$ (начерчен красным цветом), то интересующая нас область будет фигурой ограниченной этим квадратом и функцией $g$ (закрашена синим цветом). Выходит искомая вероятность равна площади синей фигуры поделенной на площадь квадрата, что равна 16.



Чтобы найти площадь синей фигуры вычтем из половины площади квадрата площадь ограниченной $g$ и $Ox$ с $-2$ по $2$. Последнее можно найти как интеграл $g$ на интервале $(-2;2)$:

$$\int_{-2}^{2}g(x) = \int_{-2}^{2} \frac{x^2}{4} = \left|\frac{x^3}{12}\right|_{-2}^{2} = \frac{2^3}{12} - \frac{(-2)^3}{12} = \frac{4}{3}.$$

Следовательно, площадь синей фигуры равна $\frac{16}{2} - \frac{4}{3} = \frac{20}{3}$. А значит ответ равен $\frac{20}{3} : 16 = \frac{5}{12}$

\end{document}